\section{Heuristics for cost scaling} \label{appendix:impl-csheuristics}
% Proofread: 1 minor edits
Goldberg has proposed a number of heuristics to improve the real-world performance of his cost scaling algorithm~\cite{Goldberg:1997}, described in \cref{sec:impl-cost-scaling}. These have been found to result in considerable real-world improvements in efficiency~\cite{Bunnagel:1998,KiralyKovacs:2012}. Note that their effectiveness depends on the problem instance. Moreover, several of the heuristics such as arc fixing and potential update are highly sensitive to parameter settings or other implementation choices.

\subsection{Potential refinement} \label{appendix:impl-csheuristics-potential-refinement}
The $\textproc{Refine}(\mathbf{x},\boldsymbol{\pi},\epsilon)$ routine is guaranteed to produce an $\epsilon$-optimal flow. However, it may also be $\epsilon'$-optimal for $\epsilon' < \epsilon$. This heuristic decreases $\epsilon$ while modifying the potentials $\boldsymbol{\pi}$, without changing the flow $\mathbf{x}$. This has been found to yield a 40\% performance improvement~\cite{Bunnagel:1998}.

\subsection{Push lookahead}
Before performing $\textproc{Push}(i,j)$, this heuristic checks whether $j$ is a deficit node ($e_j < 0$) or if $j$ has an outgoing admissible arc. If so, the \textproc{Push} operation proceeds.

Otherwise, the \textproc{Push} operation is aborted, and \textproc{Relabel} is performed instead. Were the \textproc{Push} operation to be performed, the flow would get `stuck' at node $j$, and end up being pushed back to $i$. This heuristic has been found to significantly reduce the number of \textproc{Push} operations performed~\cite{Goldberg:1997}.

\subsection{Arc fixing}
The algorithm examines a large number of arcs, sometimes unnecessarily. It can be proved that for any arc $(i,j)$ with reduced cost satisfying $\left|c^{\boldsymbol{\pi}}_{ij}\right| > 2n\epsilon$, the flow $x_{ij}$ is never again modified. These arcs can safely be \emph{fixed}: removed from the adjacency list examined by the algorithm. 

This heuristic takes things a step further, \emph{speculatively} fixing arcs whenever $\left|c^{\boldsymbol{\pi}}_{ij}\right|$ is above some threshold $\beta$. Speculatively fixed arcs may still need to have their flow updated, but this is unlikely. Consequently, the algorithm examines these arcs very infrequently, unfixing arcs found to violate reduced cost optimality conditions.

\subsection{Potential update}
The \textproc{Relabel} operation given in \cref{algo:cost-scaling-operations} updates the potential at a single node. This heuristic is based around an alternative \emph{set relabel} operation, which updates potentials at many nodes at once.

Regular relabel operations are still used, with set relabel called periodically. The optimum frequency is dependent on both the problem and implementation, but calling set relabel every $O(n)$ regular relabels has been found to be a good rule of thumb~\cite{Goldberg:1997}.

This heuristic improves performance when used on its own, but has little effect when used in conjunction with potential refinement and push lookahead described above. In fact, it may even \emph{harm} performance in some cases, although it has been found to be more beneficial the larger the network instance~\cite{Bunnagel:1998}.

\section{Test harness} \label{appendix:impl-benchmark-harness}

Running the tests manually would be unwieldly and error-prone. A test harness was developed in Python to automate the process. The configuration of the harness includes a list of implementations and datasets. Individual test cases in the configuration specify which implementation to test and what datasets to use, along with other parameters.

In addition to testing the final versions of the solvers, the harness was used to evaluate the impact of optimisations implemented throughout the project, as reported in \cref{sec:eval-optimisations}. To support this, the harness was integrated with the Git version control system~\cite{GitWWW}. Versions of an implementation can be specified by a Git commit ID.

\todo{Add extract from configuration file / example output from the harness to illustrate this?}
It was anticipated that this project would involve running a large number of experiments. The harness was therefore designed to make the process as simple as possible. Test cases can be specified in less than 10 lines. Starting the experiment is simple: the harness will check out appropriate versions of the code and build the implementations with no user intervention, returning once the results of the test are complete.

Efficiency was also a key requirement: with many experiments, it is important that they complete as fast as possible. The harness reuses intermediate computation where possible. Implementations are only checked out and built once. A cache of generated flow networks is maintained\footnotemark, so that the overhead is only incurred on the first run.
\footnotetext{Except where this could affect the results. For example, the test to measure scheduling latency described in the previous section requires the network be regenerated on each run, as it may vary slightly depending on the runtime of the algorithm.}

% Omitted, may want to mention:
% - Supports both offline (graph file) and hybrid/online (simulator) tests
% - For offline: supports both incremental and full, via snapshot solver.
% - For offline: can specify files with globs, etc.