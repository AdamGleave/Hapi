\chapter{Implementation} \label{chap:impl}

\section{Introduction}

As discussed in \S\ref{sec:intro-challenges} and \S\ref{sec:intro-related-work}, considerable research effort has been expended over the past 60 years to produce efficient flow algorithms. Developing a better general-purpose minimum-cost flow algorithm would be a considerable undertaking, more suited to a PhD than a Part II project. Rather than attempting to supplant prior work, my approach is to instead modify and extend it, in order to achieve improved performance in the special-case of networks produced by Quincy-like systems.

Two strategies seemed particularly promising. We could loosen the constraints on the problem, finding \emph{approximate} solutions, discussed in \S\ref{sec:impl-approx}. All extant minimum-cost flow algorithms are designed to find optimal solutions. But for flow schedulers, we may want to trade optimality for reduced scheduling latency. The other strategy is to solve the problem \emph{incrementally}, discussed in \S\ref{sec:impl-incremental}. The network tends to remain mostly unchanged between successive invocations of the scheduler. Suppose a task is submitted: a node will be added, along with a handful of arcs, but no other changes will take place. Significant performance improvements can be realised in cases such as these by reoptimising from the last optimal solution found.

These strategies are not solution methods in and of themselves. Rather, they suggest modifications that can be made to flow algorithms. Consequently, I will start this chapter by describing the implementation of several minimum-cost flow algorithms. Once I have described the standard algorithms implemented in this project, I will move on to discussing the two strategies in more detail. Note that algorithms vary in their suitability to the two strategies; justification will be provided where appropriate.

% TODO: Asymptotic complexity summary somewhere?

\section{Cycle cancelling algorithm}

% Should try and keep fairly consistent structure between algorithms
% Malte, Ionel seemed to think good to briefly discuss properties of the algorithm.
% You'll also want to mention this in the strategies section, though.
% It's important, so think OK to say it twice. But which will be the more detailed one?
% I think depends: if it's specific to a particular algorithm, place it there. But if it's a more general point (e.g. primal vs dual), place it in strategy

Cycle cancelling is the simplest algorithm we will consider. It has little to recommend it from a performance perspective: the original version due to Klein~\cite{Klein:1967}, the focus of this section, is exponential time in the worst case, although in practice its performance is often better than this. Variants have achieved strongly polynomial bounds~\cite{Goldberg:1989,Sokkalingam:2000}, but are still slower both theoretically and empirically than competing algorithms. However, it is a good starting point: it will help cast light on many of the techniques used in more sophisticated algorithms.

\subsection{Algorithm description}

Cycle cancelling is inspired by the negative cycle optimality conditions, given in \cref{thm:optimality-neg-cycle}.

\begin{algorithm}
    \caption{Cycle cancelling algorithm}
    \label{algo:cycle-cancelling}
    \begin{algorithmic}[1]
        \State $\mathbf{x}\gets $ result of maximum-flow algorithm \Comment{establishes x feasible}
        \While{$G_\mathbf{x}$ contains a negative cost cycle}
        \State identify negative cost cycle $W$ \Comment{e.g. using Bellman-Ford}
        \Let{$\delta$}{$\min_{(i,j) \in W} r_{ij}$}
        \State augment $\delta$ units of flow along cycle $W$
        \EndWhile{}
    \end{algorithmic}
\end{algorithm}

The algorithm is initialised with a feasible flow $\mathbf{x}$, which can be found by any maximum-flow algorithm, such as Ford-Fulkerson~\cite{FordFulkerson:1956}. The feasibility of the solution $\mathbf{x}$ is maintained throughout the algorithm, and its cost is reduced. During each iteration of the algorithm, a directed cycle of negative cost is identified in the residual network $G_\mathbf{x}$ and \emph{cancelled} by pushing the maximum possible amount of flow along it. This will cause the cycle to 'drop out' of the residual network: one or more arcs along the cycle will be saturated, and so no longer present in the residual network. The algorithm terminates when no negative cost directed cycles remain.

Note this generic version of the algorithm does not specify \emph{how} negative cycles are to be identified. I opted to use the well-known Bellman-Ford~\cite[p.~651]{CLRS:2009} algorithm for this purpose. Other, more efficient algorithms could be used. However, cycle cancelling was never going to be fast enough for my needs\footnotemark. I chose to implement it so as to have a known-working algorithm early in the project, simplifying subsequent testing. For this purpose, cycle cancelling using the Bellman-Ford algorithm is ideal.

\footnotetext{The asymptotic complexity is considerably worse than that of competing algorithms, and computational benchmarks~\cite{KiralyKovacs:2012} were also unfavourable.}

\subsection{Analysis}

\subsubsection{Correctness}

We will show that, if the algorithm terminates, it produces the correct result. \\

\begin{thm} \label{thm:cycle-cancelling-invariant}
Immediately before each iteration of the loop, $\mathbf{x}$ is a feasible solution.
\end{thm} 
\begin{proof}
For the base case, $\mathbf{x}$ must be initially feasible, by correctness of the maximum-flow algorithm used.

For the inductive case, suppose $\mathbf{x}$ is feasible immediately prior to an iteration of the loop body. The loop body pushes flow along a cycle. This maintains feasibility: the excess at the nodes along the cycle must remain zero, since any increase in the flow leaving the node is counterbalanced by an equal increase in the flow entering the node.
\end{proof}

\begin{cor}
Upon termination, $\mathbf{x}$ is a solution of the minimum-cost flow problem.
\end{cor}
\begin{proof}
By \cref{thm:cycle-cancelling-invariant}, $\mathbf{x}$ is a feasible solution upon termination. The algorithm only terminates when no negative-cost directed cycles exist. It follows by \cref{thm:optimality-neg-cycle} that $\mathbf{x}$ is optimal.
\end{proof}

\subsubsection{Termination and asymptotic complexity}

We will now show that the algorithm always terminates, and provide a bound on its running time.\\

\begin{thm} \label{thm:cycle-cancelling-termination}
The algorithm terminates within $O(mCU)$ iterations.
\end{thm}
\begin{proof}
Clearly $mCU$ is an upper bound on the cost of the initial flow. Each iteration of the algorithm identifies a negative cost cycle $w$. The cost of sending one unit of flow along this cycle is $c = \sum_{(i,j) \in w} c_{ij}$, which is strictly negative by definition. We send $\delta = \min_{(i,j) \in w} r_{ij}$ units of flow along the cycle. $\delta$ is strictly positive, otherwise the cycle would not exist in the \emph{residual} network. 

The objective function value changes by $c\delta$. By \cref{assumption:integrality}, $c$ and $\delta$ must both be integral. So as $c\delta < 0$, we have $c\delta \leq 1$.

Hence the cost decreases by at least one each iteration, and so the number of iterations is bounded by $O(mCU)$.
\end{proof}

\begin{cor} \label{corollary:cycle-cancelling-complexity}
The asymptotic complexity is $O(nm^2CU)$.
\end{cor}
\begin{proof}
Note that Bellman-Ford runs in $O(nm)$ time. Augmenting flow along the cycle is of cost linear in the length of the cycle, and so is certainly $O(m)$. Thus each iteration runs in $O(nm)$. By \cref{thm:cycle-cancelling-termination}, it follows the complexity of the algorithm is $O(nm^2CU)$.
\end{proof}

\section{Successive shortest path algorithm}

% Note this algorithm offers polynomial performance for Quincy-like graphs because it is independent of cost, whereas cycle cancelling is not
This algorithm runs in weakly polynomial time. We will show, however, that it has a strongly polynomial time bound of $O(n^2)$ for the class of flow networks produced by Quincy-like systems. It lends itself readily to an incremental implementation (see \S\ref{sec:impl-incremental}), but is inappropriate for an approximate solver.

\subsection{Algorithm description}

\begin{algorithm}
    \caption{Successive shortest path algorithm}
    \label{algo:successive-shortest-path}
    \begin{algorithmic}[1]
        \State $\mathbf{x} \gets \mathbf{0}$ and $\boldsymbol{\pi} \gets \mathbf{0}$
        \While{mass balance constraints not satisfied}
          \State choose node $s$ with $e(s) > 0$ and node $t$ with $e(t) < 0$\footnotemark
          \State solve SSSP problem from $s$ to all other nodes, in the residual network $G_{\mathbf{x}}$ with respect to the reduced costs $c^{\boldsymbol{\pi}}_{ij}$
          \State let $\mathbf{d}$ denote the vector of shortest path distances, s.t. $d_i$ is the shortest path from $s$ to $i\in V$
          \Let{$\boldsymbol{\pi}$}{$\boldsymbol{\pi} - \mathbf{d}$}
          \State let $P$ denote a shortest path from $s$ to $t$
          \Let{$\delta$}{$\min\left(e(s), -e(t), \min\left\{ r_{ij} \::\: (i,j) \in P\right\}\right)$}
          \State augment $\delta$ units of flow along path $P$
        \EndWhile
    \end{algorithmic}
\end{algorithm}

\footnotetext{Note whilst mass balance constraints are unsatisfied there must exist both a source node $s$ and sink node $t$. This is since the total sum of excesses must equal the total sum of deficits, otherwise $\sum_v b(v) \neq 0$ and the problem is infeasible.}

The successive shortest path algorithm maintains a pseudoflow (see \S\ref{sec:prep-flow-pseudo}) $\mathbf{x}$ and potentials $\boldsymbol{\pi}$ which satisfies reduced cost optimality (see \cref{thm:optimality-reduced-cost}), and attempts to attain feasibility. This is in contrast to the cycle cancelling algorithm, which maintains the feasibility strives to achieve optimality.

Each iteration of the algorithm identifies a source node $s$ and sink node $t$. The single-source shortest-path (SSSP) problem~\cite{ch.~24,CLRS:2009} is then solved from $s$. The pseudoflow $\mathbf{x}$ is updated by augmenting along a shortest path to $t$, $P$. We are limited by the minimum residual capacity of arcs along the path $P$. Moreover, we opt to maintain a non-negative supply at $s$ and demand at $t$. This restriction ensures the magnitude of the excess at each node is monotonically decreasing. We also update the potentials $\boldsymbol{\pi}$, to maintain reduced cost optimality. The algorithm terminates when no source nodes or sink nodes exist.

\subsection{Analysis}

\subsubsection{Correctness}

First, we will show that the algorithm maintains reduced cost optimality, for which we will need a number of lemmas. We will then conclude, using the invariant and the terminating condition, that the algorithm will return a solution to the minimum-cost flow problem.\\

\begin{lemma} \label{lemma:ssp-reduced-costs}
Let a pseudoflow $\mathbf{x}$ satisfy the reduced cost optimality conditions \cref{eq:optimality-reduced-cost} with respect to potentials $\boldsymbol{\pi}$. Let $\mathbf{d}$ represent the shortest path distances from a node $s \in V$ to all other nodes in the residual network $G_{\mathbf{x}}$ with respect to the reduced costs $c^{\boldsymbol{\pi}}_{ij}$. Then:
    
\begin{enumerate}[label=(\alph*)]
  \item $\mathbf{x}$ also satisfies reduced cost optimality conditions with respect to potentials $\boldsymbol{\pi}' = \boldsymbol{\pi} - \mathbf{d}$.
  \item The reduced costs, $c^{\boldsymbol{\pi}'}_{ij}$, with respect to this new potential $\boldsymbol{\pi}'$, are zero for all arcs $(i,j)$ in the shortest-path tree rooted at $s \in V$.
\end{enumerate}
\end{lemma}
\begin{proof}
See~\cite[p.~320]{Ahuja:1993}.
\end{proof}

\begin{cor} \label{cor:ssp-reduced-costs}
Let a pseudoflow $\mathbf{x}$ satisfy the reduced cost optimality conditions, with respect to some potentials $\boldsymbol{\pi}$. Let $\mathbf{x}'$ denote the pseudoflow obtained from $x$ by sending flow along a shortest path from node $s$ to some other node $k \in V$. Then $x'$ also satisfies the reduced cost optimality conditions, with respect to potentials $\boldsymbol{\pi}' = \boldsymbol{\pi} - \mathbf{d}$.
\end{cor}
\begin{proof}
By \cref{lemma:ssp-reduced-costs}(a), $(\mathbf{x},\boldsymbol{\pi'}$ satisfies the reduced cost optimality conditions.

Pushing flow along an arc $(i,j) \in G_{\mathbf{x}}$ might add its reversal $(j,i)$ in the residual network. Let $P$ be a shortest path from $s$ to $k$. By \cref{lemma:ssp-reduced-costs}(b), it follows that any arc $(i,j) \in P$ has $c^{\boldsymbol{\pi}'}_{ij} = 0$. So $c^{\boldsymbol{\pi}'}_{ji} = 0$. Thus any arcs that are added to the residual network by augmenting flow along $P$ have a zero reduced cost, and so still satisfy the reduced cost optimality conditions \cref{eq:optimality-reduced-cost}.
\end{proof}

\begin{thm} \label{thm:ssp-invariant}
Immediately before each iteration of the loop, $(\mathbf{x},\boldsymbol{\pi})$ satisfies reduced cost optimality.
\end{thm}
\begin{proof} (Induction)
    
For the base case, note $(\mathbf{0},\mathbf{0})$ satisfy reduced cost optimality. We have $G_{\boldsymbol{0}} = G$, i.e. the residual and original network are the same. Moreover, all arc costs $c_{ij}$ are non-negative (by \cref{assumption:non-negative-arc-costs}) and so the reduced costs $c^{\boldsymbol{0}}_{ij}$ are also non-negative. Thus \cref{eq:optimality-reduced-cost} holds.

Now, suppose the inductive hypothesis holds: reduced cost optimality holds immediately prior to execution of the loop body. The loop body computes a shortest path distances $s$ from a node $d$, updates $\boldsymbol{\pi}$ to become $\boldsymbol{\pi'}$ as defined in \cref{lemma:ssp-reduced-costs} and pushes flow along a shortest path from $s$ to another node, yielding a new flow $\mathbf{x}$ of the same form as $\mathbf{x'}$ in \cref{cor:ssp-reduced-costs}. It follows by \cref{cor:ssp-reduced-costs} that $(\mathbf{x},\boldsymbol{\pi})$ satisfy reduced cost optimality at the end of the loop body. Hence, the inductive hypothesis continues to hold.
\end{proof}

\begin{thm} \label{thm:ssp-correctness}
Upon termination, $\mathbf{x}$ is a solution of the minimum-cost flow problem.
\end{thm}
\begin{proof}
The algorithm terminates when the mass balance constraints \cref{eq:mass-balance} are satisfied. At this point, the solution $\mathbf{x}$ is feasible (see \S\ref{sec:prep-flow-pseudo}). 

By \cref{thm:ssp-invariant}, we know the algorithm maintains the invariant that $\mathbf{x}$ satisfies reduced cost optimality. 

It follows that $\mathbf{x}$ is both a feasible flow and optimal upon termination, so $\mathbf{x}$ is a solution of the minimum-cost flow problem.
\end{proof}

\subsubsection{Termination and asymptotic complexity}

\begin{thm} \label{thm:ssp-complexity}
Let $S(n,m,C)$ denote the time complexity of solving a single-source shortest path problem with non-negative arc costs, bounded above by $C$, over a network with $n$ nodes and $m$ arcs. Then the time complexity of successive shortest path is $O(nUS(n,m,C))$.
\end{thm}
\begin{proof}
Each iteration of the loop body (lines 3-9) decreases the excess of some node $s$ by $k$ and the deficit of some node $t$ by $k$, whilst leaving the excess/deficit of other nodes unchanged. Consequently, the total excess and total deficit are both decreased by $k$. By \cref{assumption:integrality}, $k \geq 1$. So the number of iterations is bounded by the initial total excess. But this is bounded by $nU/2 = O(nU)$. So there are $O(nU)$ iterations.

Within each iteration, the algorithm solves a single-source shortest-path problem on line 4. Since reduced cost optimality is maintained throughout the algorithm, the arc costs in the shortest-path problem are non-negative\footnotemark. Thus the cost of solving this problem is $S(n,m,C)$.
\footnotetext{Algorithms such as Djikstra which assume non-negative arc lengths are asymptotically faster than more general algorithms such as Bellman-Ford.}

The cost of line 3 is $O(n)$, line 6 $O(n)$ and line 8-9 $O(n)$ as the length of $P$ is bounded by $n-1$ (shortest path is acyclic). Certainly $S(n,m,C) = \Omega(n)$, so the cost of each iteration is $O(S(n,m,C))$.

It follows that the overall time complexity of the algorithm is $O(nUS(n,m,C))$.
\end{proof}

% TBC: Intro promises we'll derive an asymptotic complexity bound of n^2 for Quincy-style networks. I think this is wrong, and in any case we don't show it.
% TBC: Related to this, explain why capacity scaling is unimportant for our application.

\subsection{Optimisations}

% TBC: ap_big_vs_small_heap: keep all vertices in the priority queue, or just add them as needed? Performance will depend on how many vertices get explored before Djikstra quits.
% Unclear whether best to put it here, or in the evaluation section, or both?

\subsubsection{Choice of shortest-path algorithm}
The fastest known single-source shortest-path algorithms are all variants of Djikstra's algorithm~\cite[ch.~4]{Ahuja:1993}. They differ in the heap data structure used to provide the priority queue needed by Djikstra. The asymptotically fastest are Fibonacci heaps, with $S(m,n,C) = O(m + n\lg n)$. By contrast, the widely used binary heap data structure gives $S(m,n,C) = O(m\lg n)$. 

Fibonacci heap's have considerable implementation complexity, however: whilst asymptotically faster, the constant factor hidden by the asymptotic notation is much greater. Computational benchmarks have found them to be slower than binary heaps in practice, for all but the largest of graphs~\cite[p.~15]{KiralyKovacs:2012}. In any case, for the class of networks produced by Quincy-like systems, $m = O(n)$, and so the two are asymptotically equivalent. Given this, I opted for a binary heap implementation. The resulting the asymptotic complexity of my successive shortest path implementation is $O(nmU \lg n)$.

\subsubsection{Terminating Djikstra's algorithm early}

% TBC: Test this using benchmark suite and include in evaluation section? Doesn't seem like it'd be too difficult.
It is possible to modify the successive shortest path algorithm to terminate Djikstra as soon as it permanently labels a deficit node $l$. Whilst not affecting asymptotic complexity, this may considerably improve performance in practice.\\

\begin{lemma}
Recall \cref{lemma:ssp-reduced-costs}. Let us redefine:
{\normalfont
\[\boldsymbol{\pi}'_{i}=\begin{cases}
\boldsymbol{\pi}_{i}-d_{i} & \textrm{if $i$ permanently labelled}\\
\boldsymbol{\pi}_{i}-d_{l} & \textrm{otherwise}
\end{cases}\]}
The original result for (a) still holds. The result for (b) holds along the shortest path from $s$ to $l$\footnotemark.
\footnotetext{Note that this is all that is needed for the correctness of the algorithm, as this is the only path along which we augment flow.}
\end{lemma}
\begin{proof}
The original proof for the lemma (given in~\cite[p.~320]{Ahuja:1993}) uses the triangle inequality:
\begin{equation} \label{eq:ssp-reduced-costs-triangle}
d_j \leq d_i + c^{\boldsymbol{\pi}}_{ij}\:\forall(i,j)\in E_{\mathbf{x}}
\end{equation}
Terminating Djikstra's algorithm early, we only know the shortest-path distance to nodes which have been permanently labelled. But for any node $i$ not permanently labelled and node $l$ permanently labelled, the shortest path distances satisfy:
\begin{equation} \label{eq:ssp-djikstra-labelled}
d_i \geq d_l
\end{equation}
This is because were $d_i < d_l$ then $i$ must have been permanently labelled before $l$, which is a contradiction.

Now, let us define: 
\begin{equation} \label{eq:ssp-djikstra-distances}
d'_{i}=\begin{cases}
d_{i} & \text{if $i$ permanently labelled}\\
d_{l} & \text{otherwise}
\end{cases}
\end{equation}

Given \cref{eq:ssp-djikstra-labelled}, we know $\mathbf{d}'$ satisfies \cref{eq:ssp-reduced-costs-triangle}. The original proof for (a) thus still holds, as it makes no further assumptions on $\mathbf{d}'$.

As for (b), every node $i$ along the shortest path from $s$ to $l$ has been permanently labelled, and so $d'_i = d_i$. Hence the original proof still holds along this path.
\end{proof}

Any constant shift in the potential for every node will leave reduced costs unchanged. So we may equivalently redefine:

\[\boldsymbol{\pi}'_{i}=\begin{cases}
\boldsymbol{\pi}_{i}-d_i+d_l & \text{if $i$ permanently labelled}\\
\boldsymbol{\pi}_{i} & \text{otherwise}
\end{cases}\]

This is computationally more efficient, as we will not have to update the potential at any node which has not been permanently labelled (typically the majority of nodes).

\section{Relaxation algorithm} \label{sec:impl-relax}

Like the successive shortest path algorithm, the relaxation algorithm works by augmenting along shortest paths in the residual network from sources to sinks. Based upon the method of Lagrangian relaxation from mathematical optimization~\cite[ch.~16]{Ahuja:1993}\cite{Fisher:1981}, unlike the successive shortest path algorithm it updates node potentials using intermediate information as it constructs the shortest-path tree. For this reason, it performs much better empirically than the successive shortest path algorithm, and indeed is one of the fastest algorithms for some classes of flow networks~\cite{KiralyKovacs:2012}. However, its worst-case time complexity is exponential, giving it the slowest theoretical performance of the algorithms we consider.\footnotemark Like successive shortest path, it is well suited to being used in an incremental solver, but is inappropriate for an approximate solver.

\footnotetext{The generic version of cycle cancelling is also exponential time, however there are variants which are strongly polynomial. By contrast, there are no variants of relaxation achieving polynomial time, although in practice exponential runtime is not observed on realistic flow networks.}

\subsection{The relaxed integer programming problem}
%TODO: Attribution for Ahuja, et al?
Before we can describe how the relaxation algorithm works, we must understand the problem it seeks to optimise. Previously we have discussed the primal (see \S\ref{sec:prep-flow-mcf}) and dual (see \S\ref{sec:prep-flow-rc-and-dual}) version of the minimum-cost flow problem. We apply Lagrangian relaxation to the primal version to yield a relaxed problem, which is in some sense a hybrid of the primal and dual statement. We denote this formulation of the optimisation problem as $\mathrm{LR}(\boldsymbol{\pi})$.

\begin{equation} \label{eq:relax-obj-fun-excess}
\mathrm{maximise}\: w\left(\boldsymbol{\pi}\right)=\min_{x}\left[\sum_{\left(i,j\right)\in E}c_{ij}x_{ij}+\sum_{i\in V}\boldsymbol{\pi}_{i}e_{i}\right]
\end{equation}
where $\mathbf{x}$ is subject to capacity constraints:
\begin{equation} \label{eq:relax-capacity-constraints}
0\leq x_{ij}\leq u_{ij}\:\forall\left(i,j\right)\in E
\end{equation}

\begin{lemma}
An equivalent definition for $w(\boldsymbol{\pi})$ is:
\begin{equation} \label{eq:relax-obj-fun-balance}
w(\boldsymbol{\pi})=\min_{x}\left[\sum_{\left(i,j\right)\in E}c_{ij}^{\boldsymbol{\pi}}x_{ij}+\sum_{i\in V}\pi_{i}b_{i}\right]
\end{equation}
\end{lemma}
\begin{proof}
Recall the excess at node $i$ is defined as:
\[e_{i}=b_{i}+\sum_{(j,i)\in E}x_{ji}-\sum_{(i,j)\in E}x_{ij}\]
So:
\begin{align*}
w\left(\pi\right)= & \min_{x}\left[\sum_{\left(i,j\right)\in E}c_{ij}x_{ij}+\sum_{i\in V}\pi_{i}\left(b_{i}+\sum_{(j,i)\in E}x_{ji}-\sum_{(i,j)\in E}x_{ij}\right)\right]\:\mbox{substituting for }e_{i}\\
= & \min_{x}\left[\sum_{\left(i,j\right)\in E}c_{ij}x_{ij}+\sum_{i\in V}\pi_{i}b_{i}+\sum_{i\in V}\pi_{i}\sum_{(j,i)\in E}x_{ji}-\sum_{i\in V}\pi_{i}\sum_{(i,j)\in E}x_{ij}\right]\:\mbox{expanding}\\
= & \min_{x}\left[\sum_{\left(i,j\right)\in E}c_{ij}x_{ij}+\sum_{i\in V}\pi_{i}b_{i}+\sum_{(j,i)\in E}\pi_{i}x_{ji}-\sum_{(i,j)\in E}\pi_{i}x_{ij}\right]\:\mbox{double to single sum}\\
= & \min_{x}\left[\sum_{\left(i,j\right)\in E}c_{ij}x_{ij}+\sum_{i\in V}\pi_{i}b_{i}+\sum_{(i,j)\in E}\pi_{j}x_{ij}-\sum_{(i,j)\in E}\pi_{i}x_{ij}\right]\:\mbox{permuting \ensuremath{i} and \ensuremath{j} in 3rd sum}\\
= & \min_{x}\left[\sum_{\left(i,j\right)\in E}\left(c_{ij}-\pi_{i}+\pi_{j}\right)x_{ij}+\sum_{i\in V}\pi_{i}b_{i}\right]\:\mbox{factoring}\\
= & \min_{x}\left[\sum_{\left(i,j\right)\in E}c_{ij}^{\boldsymbol{\pi}}x_{ij}+\sum_{i\in V}\pi_{i}b_{i}\right]\:\mbox{substituting reduced cost}
\end{align*}
\end{proof}

We may use either form for $w(\boldsymbol{\pi})$ in the remainder of this section, depending on which is most convenient. Note \cref{eq:relax-obj-fun-excess} is expressed in terms of the arc costs and excesses ($\mathbf{x}$-dependent), whereas \cref{eq:relax-obj-fun-balance} is expressed in terms of the reduced costs ($\boldsymbol{\pi}$-dependent) and balances.

\begin{lemma} \label{lemma:relax-rc-lr-equivalence}
Let $\mathbf{x}$ be a pseudoflow and $\boldsymbol{\pi}$ node potentials. Then $\left(\mathbf{x},\boldsymbol{\pi}\right)$ satisfies reduced cost optimality conditions if and only if $\mathbf{x}$ is an optimal solution to $\mathrm{LR}(\boldsymbol{\pi})$.
\end{lemma}
\begin{proof}
To simplify the proof, we will use the complementary slackness conditions rather than the reduced cost optimality conditions. Recall the two conditions are equivalent, see \S\ref{prep:flow-optimality}.

Now, consider the form of $w(\boldsymbol{\pi})$ stated in terms of reduced cost and balances, given in \cref{eq:relax-obj-fun-balance}. Thus $\mathbf{x}$ is an optimal solution to $\mathrm{LR}(\boldsymbol{\pi})$ if and only if it minimises:
\[\sum_{\left(i,j\right)\in E}c_{ij}^{\boldsymbol{\pi}}x_{ij}+\sum_{i\in V}\pi_{i}b_{i}\]
subject to the capacity constraints given in \cref{eq:relax-capacity-constraints}.

The second sum, $\sum_{i \in V} \pi_i b_i$, is constant in $\mathbf{x}$. Thus $\mathbf{x}$ is an optimal solution to $\mathrm{LR}(\boldsymbol{\pi})$ if and only if it minimises:
\[\sum_{\left(i,j\right)\in E}c_{ij}^{\boldsymbol{\pi}}x_{ij}\]

Note the coefficients $c_{ij}^{\boldsymbol{\pi}}$ are constant in $\mathbf{x}$. Furthermore, note each term is independent of each other: by varying $x_{ij}$, only the term $c_{ij}^{\boldsymbol{\pi}}x_{ij}$ is affected\footnotemark. Thus the sum is minimised when each of its summands is minimised.
\footnotetext{Contrast this to if we were varying a node quantity, such as $\pi_i$, which would affect many arcs.}

When $c_{ij}^{\boldsymbol{\pi}}>0$, the term $c_{ij}^{\boldsymbol{\pi}}x_{ij}$ is minimised by setting $x_{ij}$ to the smallest value permitted by \cref{eq:relax-capacity-constraints}, $0$. When $c_{ij}^{\boldsymbol{\pi}}<0$, set $x_{ij}$ to the largest possible value, $u_{ij}$. When $c_{ij}^{\boldsymbol{\pi}}=0$, the choice of $x_{ij}=0$. This is the same as the complementary slackness conditions given in \cref{thm:optimality-complementary-slackness}. 

Thus complementary slackness optimality is equivalent to $\mathbf{x}$ being an optimal solution to $\mathrm{LR}(\boldsymbol{\pi})$. By equivalence of reduced cost optimality and complementary slackness optimality, the result follows.
\end{proof}

\begin{defn}
Let $z^*$ denote the optimal objective function value of the minimum-cost flow problem, that is the cost of an optimal flow.
\end{defn}
~ % force line break
\begin{lemma} \label{lemma:relax-dual-optimality}
~ % force line break
\begin{enumerate}[label=(\alph*)]
    \item For any potential vector $\boldsymbol{\pi}$, $w(\boldsymbol{\pi}) \leq z^*$.
    \item There exists node potentials $\boldsymbol{\pi}^*$ for which $w(\boldsymbol{\pi}^*) = z^*$.
\end{enumerate}
\end{lemma}
\begin{proof}
For the first part, let $\mathbf{x}^*$ be a feasible flow with objective function value $s(\mathbf{x}^*) = z^*$, that is $\mathbf{x}^*$ an optimal solution to the minimum-cost flow problem given in \cref{eq:mcf-primal-problem}.

Recall the form of $w(\boldsymbol{\pi})$ with original arc costs and excesses, given in \cref{eq:relax-obj-fun-excess}. Thus we have:
\[w(\boldsymbol{\pi})\leq\sum_{\left(i,j\right)\in E}c_{ij}x_{ij}^{*}+\sum_{i\in V}\pi_{i}e_{i}\]
by dropping the $\min_x$ and replacing $=$ with $\leq$.

Note the first sum is equal to $s(\mathbf{x}^*) = z^*$. Since $\mathbf{x}^*$ is a feasible flow, the mass balance constraints are satisfied, so:
\[e_i = 0\:\forall i \in V\]
and thus the second sum satisfies:
\[\sum_{i \in V} \pi_i e_i = 0\]

It follows that:
\[w(\boldsymbol{\pi}) \leq z^* + 0 = z^*\]

To prove the second part, let $\boldsymbol{\pi}^*$ be node potentials such that $\left(\mathbf{x}^*,\boldsymbol{\pi}^*\right)$ satisfy the reduced cost optimality conditions given in \cref{eq:optimality-reduced-cost}\footnotemark. By \cref{lemma:relax-rc-lr-equivalence}, it follows that $\mathbf{x}^*$ is an optimal solution to $\mathrm{LR}(\boldsymbol{\pi}^*)$. Thus $w(\boldsymbol{\pi}^*) = z^*$.
\footnotetext{Note such a choice of $\boldsymbol{\pi}^*$ is guaranteed to exist since $\mathbf{x}^*$ is an optimal solution to the minimum cost flow problem.}
\end{proof}

% Highlight similarity to weak dulity theorem?

\subsection{Algorithm description}

The algorithm maintains a pseudoflow $\mathbf{x}$ and node potentials $\boldsymbol{\pi}$ such that $\mathbf{x}$ is an optimal solution to $\mathrm{LR}(\boldsymbol{\pi})$. Equivalently, by \cref{lemma:relax-rc-lr-equivalence}, $\left(\mathbf{x},\boldsymbol{\pi}\right)$ satisfy reduced cost optimality.

So long as the pseudoflow $\mathbf{x}$ is not feasible, the algorithm selects a source node $s$. It then builds a tree rooted at $s$. The algorithm can perform one of two operations, described below. An operation is run as soon as its precondition is satisfied. After it is executed, the tree is destroyed and the process repeats.

The first operation is to update the potentials, increasing the value of the objective function whilst maintaining optimality. That, is $\boldsymbol{\pi}$ is updated to $\boldsymbol{\pi}'$ such that $w\left(\boldsymbol{\pi}'\right) > w\left(\boldsymbol{\pi}\right)$ and $\mathbf{x}$ is updated to $\mathbf{x}'$ such that $\mathbf{x}'$ is an optimal solution to $\mathrm{LR}(\boldsymbol{\pi}')$.

The second possible operation is to augment the flow $\mathbf{x}$, whilst leaving potentials $\boldsymbol{\pi}$ unchanged. The new flow $\mathbf{x}'$ remains an optimal solution of $\mathrm{LR}(\boldsymbol{\pi})$, and reduces the excess of one node without increasing the excess at any source nodes nor the demand at any sink nodes.

When both operations are eligible to be executed, we favour the former. The primary goal of the algorithm is thus to increase the value of the objective function, and the secondary goal is to increase feasibility (whilst leaving the objective function unchanged).

Before we can specify the preconditions for these operations, we must define two functions.

\begin{defn}[Tree excess] \label{defn:relax-tree-excess}
Let $S$ denote the set of nodes spanned by the tree. Let:
\begin{equation} \label{eq:relax-tree-excess}
e(S) = \sum_{i \in S} e_i
\end{equation}
\end{defn}

\begin{defn}[Cuts] \label{defn:relax-cuts}
A \emph{cut} of a graph $G = (V,E)$ is a partition of $V$ into two sets: $S$ and $\overline{S} = V \setminus S$. We denote the cut by $\left[S,\overline{S}\right]$. Let $\left(S,\overline{S}\right)$ and $\left(\overline{S},S\right)$ denote the set of forward and backward arcs crossing the cut, respectively. Formally:
\[\left(S,\overline{S}\right) = \left\{(i,j) \in E\::\: i \in S \land j \in \overline{S} \right\}\]
\[\left(\overline{S},S\right) = \left\{(i,j) \in E\::\: i \in \overline{S} \land j \in S \right\}\]
\end{defn}

\begin{defn} \label{defn:relax-tree-residual}
Let:
\begin{equation} \label{eq:relax-tree-residual}
r(\boldsymbol{\pi},S) = \sum_{(i,j) \in \left(S,\overline{S}\right) \land c_{ij}^{\boldsymbol{\pi}}} r_{ij}
\end{equation}
\end{defn}

\begin{remark}
The algorithm only adds arcs with zero reduced cost to the tree. Consequently, $r(\boldsymbol{\pi},S)$ represents the total residual capacity of arcs that could be added to the tree.
\end{remark}

\begin{algorithm}
    \caption{Relaxation algorithm}
    \label{algo:relaxation}
    \begin{algorithmic}[1]
        \State $\mathbf{x} \gets \mathbf{0}$ and $\boldsymbol{\pi} \gets \mathbf{0}$
        \While{network contains a source node $s$}
        \Let{$S$}{$\left\{s\right\}$}
        \State initialise $e(S)$ and $r(\boldsymbol{\pi},S)$
        \If{$e(S) > r(\boldsymbol{\pi},S)$}
        \Call{UpdatePotentials}{}
        \EndIf
        \While{$e(S) \leq r(\boldsymbol{\pi},S)$}
        \State select arc $(i,j) \in \left(S,\overline{S}\right)$ in the residual network with $c_{ij}^{\boldsymbol{\pi}}=0$
        \If{$e_j \geq 0$}
        \Let{$\mathrm{Pred}_j$}{$i$}
        \Let{$S$}{$S \cup \left\{j\right\}$}
        \State update $e(S)$ and $r(\boldsymbol{\pi},S)$
        \Else
        \State \Call{AugmentFlow}{j}
        \Break
        \EndIf
        \EndWhile
        \If{$e(S) > r(\boldsymbol{\pi},S)$}
        \Call{UpdatePotentials}{}
        \EndIf
        \EndWhile
    \end{algorithmic}
\end{algorithm}

\begin{algorithm}
    \caption{Relaxation algorithm: potential update procedure}
    \label{algo:relaxation-update-potentials}
    \begin{algorithmic}[1]
        \Require $e(S) > r(\boldsymbol{\pi},S)$
        \Statex
        \Function{UpdatePotentials}{}
        \For{every arc $(i,j) \in \left(S,\overline{S}\right)$ in the residual network with $c_{ij}^{\boldsymbol{\pi}}=0$}
        \State saturate arc $(i,j)$ by sending $r_{ij}$ units of flow
        \EndFor
        \State compute $\alpha \gets \min\left\{c_{ij}^{\boldsymbol{\pi}}\::\:(i,j)\in\left(S,\overline{S}\right)\:\mbox{and}\:r_{ij}>0\right\}$
        \For{every node $i\in S$}
        \Let{$\pi_i$}{$\pi_i + \alpha$}
        \EndFor
        \EndFunction
    \end{algorithmic}
\end{algorithm}

\begin{algorithm}
    \caption{Relaxation algorithm: flow augmentation procedure}
    \label{algo:relaxation-augment-flow}
    \begin{algorithmic}[1]
        \Require $e(S) \leq r(\boldsymbol{\pi},S)$ and $e(t) < 0$
        \Statex
        \Function{AugmentFlow}{t}
        \State chase predecessors in $\mathrm{Pred}$ starting from $t$ to find a shortest path $P$ from $s$ to $t$
        \Let{$\delta$}{$\min\left(e(s), -e(t), \min\left\{ r_{ij} \::\: (i,j) \in P\right\}\right)$}
        \State augment $\delta$ units of flow along path $P$
        \EndFunction
    \end{algorithmic}
\end{algorithm}

Figures \ref{algo:relaxation}, \ref{algo:relaxation-augment-flow} and \ref{algo:relaxation-update-potentials} formally describe the algorithm. We maintain a set $S$ of nodes spanned by the shortest path tree. The arcs in the tree are specified by the predecessor array, $\mathrm{Pred}$. Whilst we could recompute $e(S)$ and $r(\boldsymbol{\pi},S)$ when needed, it is more efficient for us to maintain them as variables, updating them when we add a node to the tree.

The outer loop on lines 2-17 of \ref{algo:relaxation} selects a source node $s$. The inner loop on lines 6-15 adds nodes and arcs to the tree. The tree consists solely of non-deficit nodes, and arcs with zero reduced cost. It is thus a shortest path tree in the residual network: the distance between any nodes in the tree is zero\footnotemark.
\footnotetext{Note there cannot be any path in the residual network with negative distance, by assumption of reduced cost optimality.}

If a deficit node $t$ is encountered, we augment the flow along the shortest path discovered from $s$ to $t$. We may terminate before this, however, if at any point $e(S) > r(\boldsymbol{\pi},S)$; in this case, the potentials are updated instead.

\subsection{Analysis}

\subsubsection{Correctness}

\begin{lemma}[Correctness of $\mathrm{UpdatePotentials}$] \label{lemma:relax-correctness-updatepotentials}
Let $e(S) > r(\boldsymbol{\pi},S)$. Let $\left(\mathbf{x},\boldsymbol{\pi}\right)$ satisfy reduced cost optimality conditions. Then, after executing $\textproc{UpdatePotentials}$, we have a new pseudoflow and node potentials $\left(\mathbf{x},\boldsymbol{\pi}\right)$ which continue to satisfy reduced cost optimality conditions, and $w(\boldsymbol{\pi}') > w(\boldsymbol{\pi})$.
\end{lemma}
\begin{proof}
Lines 2-3 saturate all zero reduced cost arcs, so that they drop out of the residual network. This maintains reduced cost optimality, as the flow on arcs with zero reduced cost is arbitrary. Moreover, it leaves $w(\boldsymbol{\pi})$ unchanged. Recall the formulation of the objective function in terms of reduced costs and balances, given in \cref{eq:relax-obj-fun-balance}:
\[w(\boldsymbol{\pi})=\min_{x}\left[\sum_{\left(i,j\right)\in E}c_{ij}^{\boldsymbol{\pi}}x_{ij}+\sum_{i\in V}\pi_{i}b_{i}\right]\]
The second sum is unchanged, as potentials are unchanged. The first sum is also unchanged, as $x'_{ij}$ differs from $x_{ij}$ only when the reduced costs $c_{ij}^{\boldsymbol{\pi}}=0$.

Note that there is now $r(\boldsymbol{\pi},S)$ more flow leaving nodes in $S$, so the tree excess is now:
\[e'(S) = e(S) - r(\boldsymbol{\pi},S)\]
By the precondition, $e'(S)$ is (strictly) positive.

After lines 2-3, all arcs in the residual network crossing the cut have positive reduced cost\footnotemark. We let $\alpha$ be the minimal such remaining reduced cost; note $\alpha$ is (strictly) positive.
\footnotetext{Since none had negative reduced cost, by assumption of reduced cost optimality on entering the procedure.}

Note by \cref{assumption:integrality}, we have $\alpha \geq 1 $ and $e'(S) \geq 1$.

We now obtain $\boldsymbol{\pi}'$ from $\boldsymbol{\pi}$ by increasing the potential by $\alpha$ for every node in the tree $S$. Recall the formulation of the objective function in terms of costs and excesses, given in \cref{eq:relax-obj-fun-excess}:
\[w\left(\boldsymbol{\pi}\right)=\min_{x}\left[\sum_{\left(i,j\right)\in E}c_{ij}x_{ij}+\sum_{i\in V}\boldsymbol{\pi}_{i}e_{i}\right]\]
The first sum is unchanged: modifying the potentials $\boldsymbol{\pi}$ does not change the original arc cost $c_ij$ or flow $x_{ij}$. For the second sum, we have:
\begin{align*}
\sum_{i\in V}\pi'_{i}e'_{i}= & \sum_{i\in S}\left(\pi_{i}+\alpha\right)e'_{i}+\sum_{i\in\overline{S}}\pi_{i}e'_{i}\\
= & \alpha\sum_{i\in S}e'_{i}+\sum_{i\in V}\pi_{i}e'_{i}\\
= & \alpha e'(S)+\sum_{i\in V}\pi_{i}e'_{i}
\end{align*}
Since $w\left(\boldsymbol{\pi}\right)$ unchanged after updating $\mathbf{x}$, it follows:
\[w\left(\boldsymbol{\pi}'\right)=w(\boldsymbol{\pi})+\alpha e'(S)\]
We have already shown $\alpha, e'(S) \geq 1$. It follows $\alpha e'(S) \geq 1$, so $w\left(\boldsymbol{\pi}'\right) > w\left(\boldsymbol{\pi}\right)$.

It remains to show that updating potentials maintains reduced cost optimality. Note that increasing the potentials of nodes in $S$ by $\alpha$ decreases the reduced cost of arcs in $\left(S,\overline{S}\right)$ by $\alpha$, increases the reduced cost of arcs in $\left(\overline{S},S\right)$ by $\alpha$ and leaves the reduced cost of other arcs unchanged.

Prior to updating potentials, all arcs in the residual network had (strictly) positive reduced costs. Consequently, increasing the reduced cost cannot violate reduced cost optimality\footnotemark.
\footnotetext{Note increasing the reduced cost from zero to a positive number could violate optimality: an arc with zero reduced cost is permitted to have a positive flow, whereas an arc with positive reduced cost is not.} Decreasing the reduced cost might, however. But before updating the potentials, $c_{ij}^{\boldsymbol{\pi}} \geq \alpha$ for all $(i,j)\in\left(S,\overline{S}\right)\:\mbox{and}\:r_{ij}>0$, hence after the potential update $c_{ij}^{\boldsymbol{\pi}'}\geq 0$ for these arcs.
\end{proof}

\begin{lemma}[Correctness of $\mathrm{AugmentFlow}$] \label{lemma:relax-correctness-augmentflow}
Let $e(S) \leq r(\boldsymbol{\pi},S)$, and $e(t) < 0$. Let $\left(\mathbf{x},\boldsymbol{\pi}\right)$ satisfy reduced cost optimality conditions. Then, after executing $\mathrm{AugmentFlow}$, we have a new pseudoflow $\mathbf{x}'$ such that $\left(\mathbf{x}',\boldsymbol{\pi}\right)$. Moreover, under $\mathbf{x}'$ the excess at node $s$ and deficit at node $t$ decreases, without changing the excess/deficit at any other node\footnotemark.
\footnotetext{The \emph{feasibility} of the solution has increased}
\end{lemma}
\begin{proof}
Line 2 finds a shortest path $P$ from the source node $s$ to a deficit node $t$. Line 3-4 then sends as much flow as possible along path $P$, subject to:
\begin{enumerate}
    \item satisfying the capacity constraints at each arc on $P$, and
    \item ensuring $e(s) \geq 0$ and $e(t) \leq 0$\footnotemark.
\end{enumerate}
The restriction on $e(s)$ and $e(t)$ ensures we improve the feasibility of the solution: this operation monotonically decreases the excess at source nodes and the deficit at sink nodes\footnotemark.
\footnotetext{If we were to allow the algorithm to 'overshoot' and turn $s$ into a deficit node or $t$ into an excess, we would not be guaranteed to be improving the feasibility of the solution.}

Since we send an equal amount of flow on each arc in $P$, the excess at nodes other than the start and end of path $P$ - $s$ and $t$ - are unchanged.
\end{proof}

\begin{thm}[Correctness] \label{thm:relax-correctness}
Upon termination, $\mathbf{x}$ is a solution of the minimum-cost flow problem.
\end{thm}
\begin{proof}
For the same reason given in \cref{thm:ssp-invariant}, the initial values of $\left(\mathbf{x},\boldsymbol{\pi}\right)$ satisfy reduced cost optimality conditions. $\mathrm{AugmentFlows}$ and $\textproc{UpdatePotentials}$ are only invoked when their preconditions are satisfied, and so by \cref{lemma:relax-correctness-augmentflows,lemma:relax-correctness-updatepotentials} it follows that reduced cost optimality is maintained. Moreover, the main algorithm given in \cref{algo:relaxation} does not update $\mathbf{x}$ or $\boldsymbol{\pi}$ except via calls to $\mathrm{AugmentFlows}$ and $\textproc{UpdatePotentials}$. So, reduced cost optimality is maintained as an invariant throughout the algorithm.

The algorithm terminates when the mass balance constraints are satisfied, so $\mathbf{x}$ is feasible. Thus upon termination, $\mathbf{x}$ is feasible and satisfies reduced cost optimality conditions. So $\mathbf{x}$ is a solution to the minimum-cost flow problem. 
\end{proof}

\subsubsection{Termination and asymptotic complexity}

\begin{lemma} \label{lemma:relax-complexity-updatepotentials}
The time complexity of $\textproc{UpdatePotentials}$ is $O(m)$.
\end{lemma}
\begin{proof}
The for loop on lines 2-4 of \cref{algo:relaxation-update-potentials} iterates over $O(m)$ arcs. Saturating an arc is an $O(1)$ cost, so this contributes an $O(m)$ cost.

Computing $\alpha$ on line 5 is an $O(m)$ cost, since it involves iterating over $O(m)$ arcs\footnotemark.

Updating potentials on lines 6-8 is an $O(n)$ cost, as the number of nodes in $S$ is $O(n)$.

The overall time complexity is thus $O(m)$.
\end{proof}

\begin{lemma} \label{lemma:relax-complexity-augmentflow}
The time complexity of $\textproc{AugmentFlow}$ is $O(m)$.
\end{lemma}
\begin{proof}
The shortest path $P$ must be acyclic, and so its length is bounded by $n-1=O(n)$. The operations on line 2, 3 and 4 are all linear in the length of $P$, thus the overall time complexity is $O(n)$.
\end{proof}

\begin{lemma} \label{lemma:relax-iterations} 
The following properties hold:
\begin{enumerate}[label=(\alph*)]
    \item $\textproc{UpdatePotentials}$ is called at most $mCU$ times.
    \item $\textproc{AugmentFlow}$ is called at most $nU$ times in between calls to $\textproc{UpdatePotentials}$.
    \item There are at most $nmCU^2$ iterations of the outer loop on lines 2-18 of \cref{algo:relaxation}.
\end{enumerate}
\end{lemma}
\begin{proof}
By \cref{lemma:relax-dual-optimality}, $w\left(\boldsymbol{\pi}\right) \leq z^*$. A feasible flow can have a cost of at most $mCU$. By \cref{lemma:relax-correctness-updatepotentials} and \cref{assumption:integrality}, $w\left(\boldsymbol{\pi}\right)$ increases by at least one each time $\textproc{UpdatePotentials}$ is called. Moreover, $w\left(\boldsymbol{\pi}\right)$ never decreases: $\boldsymbol{\pi}$ is only modified by $\textproc{UpdatePotentials}$. Thus $\textproc{UpdatePotentials}$ is called at most $mCU$ times.

As for $\textproc{AugmentFlow}$, by \cref{lemma:relax-correctness-augmentflow} and \cref{assumption:integrality}, each call to $\textproc{AugmentFlow}$ decreases the total excess by at least one unit. The total excess in a network can be at most $nU$. The only part of the algorithm which may \emph{increase} the excess at nodes is lines 2-3 of $\textproc{UpdatePotentials}$. Thus, in between calls to $\textproc{UpdatePotentials}$, at most $nU$ calls to $\textproc{AugmentFlow}$ may be made.

For a bound on the total number of iterations, note each iteration ends after calling either $\textproc{AugmentFlow}$ on line 12 or $\textproc{UpdatePotentials}$ on line 16\footnotemark. Thus the number of iterations is bounded by the number of possible times either of these functions are called. By part (a), we know there are at most $mCU$ calls to $\textproc{UpdatePotentials}$. This together with part (b) implies there are at most $nmCU^2$ calls to $\textproc{AugmentFlow}$. This gives a bound on the number of iterations.
\footnotetext{Of course we might also call $\textproc{UpdatePotentials}$ on line 4, but there is no guarantee this will take place and so cannot form part of the complexity analysis.}
\end{proof}

\begin{thm} The algorithm runs in $O(nm^2CU^2)$ time.
\end{thm}
\begin{proof}
We will now analyse the algorithm as a whole. Initialisation on line 1 of \cref{algo:relaxation} has cost $O(m)$. The loop body on lines 3-17, excluding the inner loop on lines 6-15 and function calls, has cost $O(1)$: initialising $S$, $e(S)$ and $r(\boldsymbol{\pi},S)$ are all constant cost.

Now, let us consider the execution of the inner loop body on lines 7-14. In order to update $r(\boldsymbol{\pi},S)$, we must scan over all adjacencies when we add a node $j$ to $S$. This dominates the other costs of lines 8-10, since updating the predecessor on line 9 and inserting an element in a set on line 10 are both constant cost. Note it is natural for us to also maintain a queue of arcs $(i,j) \in \left(S,\overline{S}\right)$ in the residual network with $c_{ij}^{\boldsymbol{\pi}}=0$: this does not increase the asymptotic complexity of lines 8-10, since we are already doing much the same work when updating $r$. Given this queue, line 7 has cost $O(1)$. Thus the inner loop body has complexity, excluding procedure calls, of $O\left(\left|\mathrm{Adj}(j)\right|\right)$\footnotemark.
\footnotetext{Note we must include both \emph{incoming} and \emph{outgoing} arcs. Outgoing arcs need to be added to the sum in $r$ and added to the queue, whereas incoming arcs need to be subtracted from the sum in $r$ and removed from the queue. Summing the adjacency lists is thus bounded by $2m$ and not $m$, but this is still $O(m)$.}

Note each iteration of this loop adds a new node to $S$ on each iteration, and so is executed at most once per node. It follows that the total complexity of the inner loop on lines 6-15, excluding procedure calls, is $O(m)$. This loop thus dominates the other costs on lines 3-17, so the loop body overall has a cost of $O(m)$.

To deduce the overall time complexity, we will need the bounds proved in \cref{lemma:relax-iterations}. $\textproc{UpdatePotentials}$ is called $O(mCU)$ times and has cost $O(m)$, so contributes a total cost of $O(m^2CU)$. $\textproc{AugmentFlow}$ is called $O(nmCU^2)$ times and has cost $O(n)$, so contributes a total cost of $O(n^2mCU^2)$. Finally, the body of the while loop on lines 3-17 of \cref{algo:relaxation} is executed $O(nmCU^2)$ times. Excluding the cost of the procedure calls (which we have considered separately), each iteration costs $O(m)$, so this contributes a cost of $O(nm^2CU^2)$. This dominates the other costs, and is thus the overall time complexity of the algorithm.
\end{proof}

\section{Cost scaling algorithm}

% SOMEDAY: If you run a benchmark of reference implementations, cite it here.
This method is due to Goldberg and Tarjan~\cite{Goldberg:1987}. It runs in weakly polynomial time, with variants offering strongly polynomial bounds. It is one of the most efficient solution methods, both in theory and practice\footnotemark. Cost scaling works by successive approximation, and so it lends itself readily to an approximate solver (see \S\ref{sec:impl-approx}). It is inappropriate for an incremental solver, however.
\footnotetext{In computational benchmarks such as~\cite{KiralyKovacs:2012}, cost scaling is the fastest algorithm for many classes of network. It is occasionally beaten by other algorithms, however its performance still remains competitive in these cases. This makes it more robust than some alternative solution methods, such as relaxation (see \S\ref{sec:impl-relax}), which sometimes outperform it but can in other cases be slower by orders of magnitude.}

The algorithm maintains a feasible flow. The flow need not be optimal, but the error is bounded using the notion of $\epsilon$-optimality. Each iteration of the algorithm refines the solution, halving the error bound, until optimality is achieved. In the next section, the notion of $\epsilon$-optimality is introduced. Afterwards, the general solution method is described. Specific variants are considered next. Finally, techniques to improve the practical performance of the method are considered.

\subsection{$\epsilon$-optimality}

The notion of $\epsilon$-optimality is a relaxation of the reduced cost optimality conditions given in \cref{thm:optimality-reduced-cost}.

\begin{defn}[$\epsilon$-optimality] \label{defn:epsilon-optimality}
Let $\epsilon \geq 0$. A pseudoflow $\mathbf{x}$ is $\epsilon$-optimal with respect to node potentials $\boldsymbol{\pi}$ if the reduced cost of each arc in the residual network $G_\mathbf{x}$ is greater than $-\epsilon$:
\begin{equation} \label{eq:epsilon-optimality}
c^{\boldsymbol{\pi}}_{ij} \geq -\epsilon\:\forall (i,j) \in E_{\mathbf{x}}
\end{equation}
\end{defn}

\begin{defn} 
An edge $(i,j) \in E_{\mathbf{x}}$ is said to be \emph{in-kilter} if it satisfies the condition in \cref{eq:epsilon-optimality}.\\
\end{defn}

\begin{remark}
Note that $0$-optimality is equivalent to the conditions of reduced cost optimality. We can, however, prove a stronger result.\\
\end{remark}

\begin{thm}
Let $0 \leq \epsilon < 1/n$, and $\mathbf{x}$ an $\epsilon$-optimal feasible flow. Then $\mathbf{x}$ is optimal.
\end{thm}
\begin{proof}
Let $C$ be a simple cycle in $G_\mathbf{x}$. $C$ has at most $n$ arcs. By \cref{defn:epsilon-optimality}, for each arc $(i,j) \in C$ we have $c^{\boldsymbol{\pi}}_{ij} \geq -\epsilon$. It follows the cost of $C$ is $\geq -n\epsilon$.

By assumption on $\epsilon$, we have $n\epsilon > -1$. By \cref{assumption:integrality}, we can deduce that the cost of $C$ is $\geq 0$. The result follows from the negative cycle optimality conditions given in \cref{thm:optimality-neg-cycle}.
\end{proof}

\subsection{Generic algorithm}

% Active vertex = excess
% Admissible arc = negative reduced cost

\subsubsection{Description}

% Difference in presentation: they consider undirected edges, specify data structure.
% I'd like to keep presentation the same as in rest of chapter, not have to deal with
% forward and backwards edges. But this may not be possible...

\begin{algorithm}
\begin{algorithmic}[1]
    \Let{$\epsilon$}{$C$} \Comment{where $C$ is largest arc cost, see \S\ref{sec:prep-flow-complexity}}
    \Let{$\mathbf{x}$}{result of maximum-flow algorithm} \Comment{establishes $\mathbf{x}$ feasible}
    \Let{$\boldsymbol{\pi}$}{$\mathbf{0}$}
    \While{$\epsilon \geq 1/n$} \Comment{loop until $\mathbf{x}$ is optimal}
        \Let{$\epsilon$}{$\epsilon/2$}
        \State \Call{Refine}{$\mathbf{x}$,$\boldsymbol{\pi}$,$\epsilon$}
    \EndWhile
\end{algorithmic}
\caption{Cost scaling algorithm}
\label{algo:cost-scaling}
\end{algorithm}

\begin{algorithm}
    \begin{algorithmic}[1]
        \Function{Refine}{$\mathbf{x}$,$\boldsymbol{\pi}$,$\epsilon$}
            \For{every arc $(i,j) \in E$} \Comment{initialisation}
                \If{$c^{\boldsymbol{\pi}}_{ij} > 0$} $x_{ij} \gets 0$ \EndIf
                \If{$c^{\boldsymbol{\pi}}_{ij} < 0$} $x_{ij} \gets u_{ij}$ \EndIf
            \EndFor
            \While{mass balance constraints not satisfied} \Comment{main loop}
                \State select an excess vertex $s$
                \If{$\exists (i,j) \in E_{\mathbf{x}} \cdot c^{\boldsymbol{\pi}}_{ij} < 0$} \Call{Push}{$i$,$j$}
                \Else \enspace\Call{Relabel}{$i$,$\epsilon$}
                \EndIf
            \EndWhile
        \EndFunction
    \end{algorithmic}
    \caption{Generic \textproc{Refine} subroutine}
    \label{algo:cost-scaling-generic-refine}
\end{algorithm}

\begin{algorithm}
\begin{algorithmic}[1]
    \Require $e_i > 0$, $r_{ij} > 0$ and $c^{\boldsymbol{\pi}}_{ij} < 0$
    \Function{Push}{$i$,$j$}
        \State send $\min\left(e_i, r_{ij}\right)$ units of flow from $i$ to $j$
    \EndFunction    
    \setcounter{ALG@line}{0}
    \Statex
    \Require $e_i > 0$ and $\forall j \in V\cdot r_{ij} > 0 \implies c^{\boldsymbol{\pi}}_{ij} \geq 0$
    \Function{Relabel}{$i$,$\epsilon$}
        \Let{$\boldsymbol{\pi}_i$}{$\min \left\{\boldsymbol{\pi}_j + c_{ij} + \epsilon \::\: (i,j) \in E_{\mathbf{x}}\right\}$}
    \EndFunction
\end{algorithmic}
\caption{Push and relabel: the basic operations}
\label{algo:cost-scaling-operations}
\end{algorithm}

% High-level description. Initialize to max-flow. Run Refine routine repeatedly, halving epsilon each time. Can give time complexity.
% Refine subroutine: input (x,pi) => (x,pi) with halved epsilon optimality
% Sets flow on every out-of-kilter edge to upper bound. Makes pseudoflow 0-optimal. But we want a flow!
% PUSH/RELABEL operations preserve epsilson-optimality, and move towards feasibility.

Pseudocode. Explanation.

\subsubsection{Correctness}

% Correctness lemma for push, relabel
% Lemma: one of the preconditions will always hold
% Theorem: generic refine subroutine correct
Proofs.

\subsubsection{Complexity}

%% Asymptotic complexity analysis
% Lemma: simple path in residual networks from source to sink vertex
% Lemma: potential of vertex increases by at most 3n*epsilon during execution of Refine
% Lemma: number of relabels during an execution of Refine is at most 3n^2. Immediate from above two lemmas.
% Lemma: number of saturating push operations is at most 5nm.
% Lemma: immediately after a relabel has been applied to a vertex v, there are no admissible edges entering v
% Corollary: admissible graph is a DAG
% Sidenote: be careful what is meant by 'refine' subroutine, otherwise this lemma doesn't hold
% Lemma: number of non-saturating push operatiosn in an execution of refine is O(n^2m)
% Theorem: O(n^2m) basic operations, immediate from above lemmas.

State, without proof; reference relevant papers.

\subsection{Algorithm implementations}

\subsubsection{Sequential implementation}

% Running time: sequential implementation is O(n^2m) using naive approach. 'Wave' approach is O(n^3). Can improve to O(nm lg n) with dynamic trees. But this isn't actually used ever?!
% Data structures: list of incident undirected edges, a current edge, ...

% Sequential implementation: set Q containing vertices with positive balance
% Proof of algorithmic complexity actually not too bad, could give it perhaps
% FIFO ordering may improve complexity: possibly useful aside

Naive approach. Aside: FIFO think gives a better bound.

\subsubsection{Wave implementation}

% Wave method: maintains list L of all vertices, topologically ordered with respect to admissible graph
% Lemma: number of passes over vertex list is O(n^2)
% Theorem: wave subroutine runs in O(n^3) time. Also not too bad.
% Definitely worth mentioning these results, since you can show wave method is actually slower in practice.

Asymptotically better. But not for flow scheduling networks. Actually slower; reference benchmark.

\subsection{Improving performance}

\subsubsection{Heuristics}

% Heuristics: can be very brief, but must mention them.

Explain briefly. Reference relevant paper. Explain why you opted not to implement them.

\subsubsection{Optimisations}

% Optimisations: changing scaling factor, maintaining integrality, wave vs sequential

\section{Approximation algorithms} \label{sec:impl-approx}

\subsection{Choice of algorithm} \label{sec:impl-approx-choice}

Why Goldberg's cost scaling. Iterative, so readily amenable to approximate solutions. So are others: e.g. cycle cancelling. But Goldberg is fastest: forward reference to benchmarks you've run, or just Kiraly \& Kovacs paper.

\subsection{An iterative approximation}

Approximate algorithm yielded by ending iteration before optimality reached. We have a choice of termination condition.

No way to go from $\epsilon$-optimality to measure of accuracy. So must adopt heuristic approaches.

\subsubsection{Convergence based on cost change}

\subsubsection{Convergence based on task allocations}

\subsubsection{Hybrid schemes}

My code allows the two to be readily combined to produce different policies.

\section{Incremental algorithms} \label{sec:impl-incremental}

Motivation. Most of the time the graph changes only a small amount. Existing approaches recompute from scratch. Can we re-use information?

\subsection{Choice of algorithms}

Explain why primal methods such as Goldberg's work poorly: require solution is feasible at all times. Dual methods better. Note these are slower when run on graph from scratch.

\subsection{Dual algorithms}

Explain how an incremental solver can be built from (any) dual algorithm. Show how reduced cost optimality conditions can be preserved, whatever change occurs.

\section{Input and output}

% Drop this section if short of space, or punt to appendix. Optimisations probably the most useful/interesting ones to talk about.

Loading networks, exporting results. Neither interesting nor impressive, but it is a necessary part of the project. Keep it brief. (Or cut it entirely?)

\subsection{DIMACS}

Representation for network, and network solution. Justify why: standard, widely used, readily available test data, integration with Firmament. 

\subsection{Incremental extension}

Why DIMACS isn't suitable for incremental problem: reloading the graph and computing diff prohibitive cost. 

\subsection{Parser optimisations}

Ignore zero-capacity arcs. Entirely trivial, but did produce noticeable performance improvement. Maybe worth mentioning.

Ignoring duplicates. Necessary for correctness. Slightly more interesting: checking whether an arc is present is slow in some data structures (where the corresponding algorithm doesn't require manual lookup). Parser maintains a bitmap instead to check this quickly irrespective of data structure. Considerable performance improvement. 

\subsection{Scheduling tasks}

Show how solver can achieve original task, by integrating into a cluster scheduler. Choice of Firmament: can be briefly justified, but need not be at all (no real alternatives).

\section{Benchmark suite} \label{sec:impl-benchmark}

% TODO: Maybe include modifications to Firmament in this or a separate section

Motivation. Automate common task. Improve quality of data collection: can run experiment many times with different parameters, no possibility of human error. 

\subsubsection{Architecture}

Specify different implementations by Git 'treeish', and path. Will checkout and build automatically. 

Test cases specify implementations required, I/O parameters, number of iterations.

Output CSV file with statistics.

\subsubsection{Testing on full networks}

Non-incremental case.

\subsubsection{Testing on incremental changes}

Can test two incremental versions head-to-head. Alternately, can test an incremental version compared to applying the diff and running a full solver from scratch.