\chapter{Implementation} \label{chap:impl}

\section{Approximation algorithms} \label{sec:impl-approx}

Motivation. Do we need optimality? Schedulers widely used in industry far from optimal: e.g. reference Hadoop's FIFO scheme. Trade-off: quality of solution, speed of scheduler. Optimality probably not justified.

\subsection{Choice of algorithm} \label{sec:impl-approx-choice}

Why Goldberg's cost scaling. Iterative, so readily amenable to approximate solutions. So are others: e.g. cycle cancelling. But Goldberg is fastest: forward reference to benchmarks you've run, or just Kiraly \& Kovacs paper.

\subsection{Goldberg's cost scaling} \label{sec:impl-approx-cs}

Describe the algorithm. Can be brief: it's not work you've done, after all. But make sure to communicate how complicated the algorithm is.

\subsubsection{Algorithm description}

Brief summary.

\subsubsubsection{Data structures}

\subsubsubsection{Push operation}

\subsubsubsection{Relabel operation}

\subsubsubsection{Complete algorithm}

\subsubsection{Analysis}

Cite paper giving proof of correctness. State, with citation, time and space complexity achieved.

\subsubsection{Heuristics}

Maybe. Only if you have time to implement any!

\subsubsection{Optimisations}

e.g. vertex vs wave

\subsection{An iterative approximation}

Approximate algorithm yielded by ending iteration before optimality reached. We have a choice of termination condition.

\subsubsection{Challenges} 

No way to go from $\epsilon$-optimality to measure of accuracy. So must adopt heuristic approaches.

\subsubsection{Convergence based on cost change}

\subsubsection{Convergence based on task allocations}

\subsubsection{Hybrid schemes}

My code allows the two to be readily combined to produce different policies.

\section{Incremental algorithms}

Motivation. Most of the time the graph changes only a small amount. Existing approaches recompute from scratch. Can we re-use information?

\subsection{Choice of algorithm}

Explain why primal methods such as Goldberg's work poorly: require solution is feasible at all times. Dual methods better. Note these are slower when run on graph from scratch.

\subsection{Dual algorithms}

Explain how an incremental solver can be built from (any) dual algorithm. Show how reduced cost optimality conditions can be preserved, whatever change occurs.

\subsection{Augmenting path}

\subsubsection{Algorithm description}

Brief summary.

\subsubsection{Analysis}

Cite paper giving proof of correctness. State, with citation, time and space complexity achieved.

\subsubsection{Optimisations}

\section{Input \& output}

Loading networks, exporting results. Neither interesting nor impressive, but it is a necessary part of the project. Keep it brief. (Or cut it entirely?)

\subsection{DIMACS}

Representation for network, and network solution. Justify why: standard, widely used, readily available test data, integration with Firmament. 

\subsection{Incremental extension}

Why DIMACS isn't suitable for incremental problem: reloading the graph and computing diff prohibitive cost. 

\subsection{Parser optimisations}

Ignore zero-capacity arcs. Entirely trivial, but did produce noticeable performance improvement. Maybe worth mentioning.

Ignoring duplicates. Necessary for correctness. Slightly more interesting: checking whether an arc is present is slow in some data structures (where the corresponding algorithm doesn't require manual lookup). Parser maintains a bitmap instead to check this quickly irrespective of data structure. Considerable performance improvement. 

\subsection{Scheduling tasks}

Show how solver can achieve original task, by integrating into a cluster scheduler. Justify the choice of Firmament (open source, VIP support). Keep it brief, it's not difficult to do.

\subsection{Benchmark suite} \label{sec:impl-benchmark}

Motivation. Automate common task. Improve quality of data collection: can run experiment many times with different parameters, no possibility of human error. 

\subsubsection{Architecture}

Specify different implementations by Git 'treeish', and path. Will checkout and build automatically. 

Test cases specify implementations required, I/O parameters, number of iterations.

Output CSV file with statistics.

\subsubsection{Testing on full networks}

Non-incremental case.

\subsubsection{Testing on incremental changes}

Can test two incremental versions head-to-head. Alternately, can test an incremental version compared to applying the diff and running a full solver from scratch.